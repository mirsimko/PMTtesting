\documentclass[a4paper,10pt]{article}
\usepackage[utf8]{inputenc}
\usepackage{amsmath}
\usepackage{amsfonts}
\usepackage{breakurl}
\usepackage{indentfirst}
\usepackage{graphicx}
\usepackage{enumerate}
\usepackage[breaklinks, hidelinks]{hyperref}
\usepackage{url}
\usepackage{booktabs}

\newcommand{\vect}[1]{\ensuremath{\boldsymbol{#1}}}
\newcommand{\ddd}{\ensuremath{\,\mathrm{d}}}
\newcommand{\dd}{\mathrm{d}}
\newcommand{\Var}{\mathrm{Var}}
\newcommand{\pravYi}{P(Y_i|\alpha, \beta, \epsilon_i)}
\newcommand{\pravY}{P(\boldsymbol{Y}|\alpha, \beta, \epsilon_i)}
\newcommand{\nsigma}{\frac{N}{\sigma^2}}
\newcommand{\arpxsq}{\overline{(X^2)}}
\newcommand{\sigman}{\frac{\sigma^2}{N}}
\newcommand{\jmenovatel}{ \arpxsq - \overline{X}^2}
\usepackage{amsmath}
%opening
\title{ZDC voltages}
\author{Miroslav Šimko, Lukáš Kramárik, Jan Vaněk}
\date{}

\begin{document}

\maketitle

The gain on photomultipliers follows the power law dependence
\begin{equation}
G = \left(\frac{U}{U_0} \right)^X
\end{equation}
with the voltage $U$ where $G$ is the measured gain. The coefficients $U_0$ and $X$ have to be extracted from measurement.



\begin{table}[htb] 
\caption{Calculated voltages from the position of the single neutron peak and ratios 
between ZDC towers}
\label{uncorected}
\begin{center}
\begin{tabular}{lccccc}
 \toprule
 &$U_\text{current}$[V]&Single n pos.&gain ratio&desired gain ratio&$U_\text{result}$[V]\\
\midrule
 East&2471&38.33&0.689&0.6&2661\\
     &2779&38.33&0.228&0.3&3300\\
     &2353&38.33&0.083&0.1&2735\\
 \midrule
West&2532&43.56&0.665&0.6&2667\\
    &2643&43.56&0.278&0.3&2905\\
    &2671&43.56&0.057&0.1&3294\\
 \bottomrule
\end{tabular}
\end{center}
\end{table}

\end{document}



